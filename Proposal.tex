\documentclass[12pt]{article}
\title{Research Proposal}
\author{Michael Berelowitz - University of Connecticut}
\date{October 10, 2022}
\usepackage{geometry}
\geometry{legalpaper,  margin=1in}
\usepackage{setspace}
\onehalfspacing
\begin{document}
\maketitle

\section{Introduction}
My research will be an examination of home-field advantage in Major League Baseball (MLB), with a special focus on the potentially differing impact of said advantage during the pandemic when crowds were not at the stadiums. This is a topic that has already been extensively researched, and I am looking to see how my results will compare to those of studies that have already been done.

\section{Specific Aims}
There are two main research questions that I will be trying to answer. They are:
\begin{itemize}
\item Is there statistically significant evidence of home-field advantage in Major League Baseball games?
\item Was there a difference in the impact of home-field advantage during the 2020 season, when the stadiums did not have fans?
\end{itemize}

My hypotheses are that there is statistically significant home-field advantage in the MLB, and that this advantage was diminished during the 2020 season due to the lack of fans at games.

The reason I am studying these questions is because of the impact the idea of "home-field advantage" has on predictive analysis of sports games. The idea of such an advantage for the home team is reflected in betting odds, as well as pundits when they make pre-game predictions. Hopefully, this study will provide a concrete answer about the true impact of home-field advantage on the outcome of games.

\section{Data Description}
There are two datasets I will be using in order to conduct my research. The first is the game logs for all MLB games between 2012-2016 (five recent seasons that were the most easily attainable data). The sample size for this dataset is 12,148 games. There are 161 variables in this dataset, but the main variables of interest will be the scores of the games. This means I will mainly be looking at two variables: the visiting team's score and the home team's score. However, other variables, such as number of hits for each team or other in-game statistics may also factor into my analysis. This dataset was obtained from the website "data.world", who created the game logs from data that was originally compiled by "Retrosheet."

The second dataset I will be using is the game logs for all games in the 2020 MLB season (the season during the COVID pandemic when there were no fans in the stands). The sample size for this dataset is 898 games. As with the other dataset, I will mainly be focusing on the visiting and home team's respective scores, but may also consider other basic statistics in my analysis. This dataset was obtained from "Retrosheet."

\section{Research Methods}
My general plan for my research is that for each dataset, I will run a two-variable t-test, using the home and visiting team's scores as samples, to see if the average number of runs scored by home teams is significantly greater than the average number of runs scored by away teams. I will also calculate the percentage of games won by the home team and run a one-proportion z-test to see if the proportion of games won by the home team is significantly greater than 50%.

As for the comparison between the 2020 season and the rest of the seasons, I will run a two-proportion z-test to determine whether the percentage of games won by the home team was significantly higher in the normal seasons than it was during the COVID season.

\section{Discussion}
My expectation is that I will find statiscally significant evidence of the existance of home field advantage. This prediction is based on my general knowledge of sports and an eyeballing of the data. Looking at MLB standings, teams almost always have a better record at home than away, and it is commonly accepted in sports circles that home-field advantage exists. That being said, it is possible that my results won't support this conclusion, which would contradict most of the existing research.

I also expect to find a significant difference in the impact of home-field advantage for the 2020 season, when there were no fans in the stands. However, if the study does produce the results I expect, it would be contradictory to some existing research, such as a study in the International Journal of Sport Finance which found no significant difference in home-field advantage between the 2019 and 2020 MLB seasons (Losak).

\section{Conclusion}
The purpose of this research is to test commonly-accepted claims that are discussed in sports. One of the most common is the assumption that the home team in professional sports games has an inherent advantage. While I do believe this to be true, the only way to confirm it is to put it to the test and analyze it statiscally. That is exactly what this project will do.

\pagebreak
\begin{center}
References
\end{center}

Risser, Michael S.; Gray, Blake R.; and Kelly, Ryan A., "Impact of Home Field Advantage: Analyzed Across Three Professional Sports" (2018). Student Publications. 611. \hfill \break

Losak, Jeremy, and Joseph Sabel. “Baseball Home Field Advantage without Fans in the Stands.” International Journal of Sport Finance, vol. 16, no. 3, Aug. 2021, https://doi.org/10.32731/ijsf/163.082021.04. \hfill \break

\begin{center}
Raw Data Sources
\end{center}

https://data.world/dataquest/mlb-game-logs \hfill \break

https://www.retrosheet.org/gamelogs/index.html



\end{document}